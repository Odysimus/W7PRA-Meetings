\documentclass[11pt]{beamer}
\usepackage[utf8]{inputenc}
\usepackage[T1]{fontenc}
\usepackage{lmodern}
\usepackage{amsmath}
\usepackage{amsfonts}
\usepackage{amssymb}
\usepackage{graphicx}
\usetheme{Luebeck}
\usecolortheme{albatross}
\begin{document}
	\author{Anthony Odenthal - KE7OSN}
	\title{An Introduction to Emergency Communications}
	\subtitle{A practical guide to being able to help}
	%\logo{}
	%\institute{}
	\date{21 May 2018}
	%\subject{}
	%\setbeamercovered{transparent}
	%\setbeamertemplate{navigation symbols}{}
\AtBeginSection[]
{
	\begin{frame}
	\frametitle{Table of Contents}
	\tableofcontents[currentsection]
	\end{frame}
}

\begin{frame}[plain]
	\maketitle
\end{frame}

\begin{frame}{Table of Contents}
\tableofcontents
\end{frame}

\section{Introduction}

\begin{frame}
\frametitle{Preface}
Many hams talk about helping in an emergency but many don't ever think about what that actually means.
\end{frame}

\begin{frame}{Don't make things worse}
You might not be able to help, but you can always get in the way. But with a little preparation you can get on track to being an asset rather than a liability.
\end{frame}

\begin{frame}{It could be you}
At the end of the day remember that it could be \textbf{your} life that gets saved by knowing how to communicate with the right people.
\end{frame}

%\begin{frame}{Story Time}
%This is the story of a backpacker that broke their leg, was able to call for help on CW and get rescued...
%\end{frame}

\section{Who to talk to?}

\begin{frame}{Hams}
The people you should be best prepared to talk with are other hams.
\end{frame}

\begin{frame}{Hams}
The people you should be best prepared to talk with are other hams. \\
Make sure you know who you should be talking to, and how to talk to them.
\end{frame}

\begin{frame}{Hams}
The people you should be best prepared to talk with are other hams. \\
Make sure you know who you should be talking to, and how to talk to them. \\
Ask around of your local ARES or CERT for their frequencies and net operators.
\end{frame}

\begin{frame}{Hams cont.}
Remember that no one radio or band will let you talk to everyone
\begin{description}
	\item[UHF/VHF] Good for short distance, line of sight, very reliable but limited range
	\item[HF w/ NVIS] Good for going further than UHF/VHF and covering the region. Cover a wider area than UHF/VHF but not as far as skip, less reliable than UHF/VHF.
	\item[HF w/ skip] Good for long distance, needs an antenna high enough, more fickle than UHF/VHF, tend to skip over nearby stations
\end{description}
\end{frame}

\begin{frame}{Hams cont.}
Decide who you want to be able to talk to and figure out what you need to be able to do that with. Setting up a large HF station to talk down the street would be overkill, and an HT is going to have a tough time getting across the country.
\end{frame}

\begin{frame}{CERT}
	\begin{description}
		\item[CERT] Community Emergency Response Team
	\end{description}
Is a FEMA organized program to train people to respond to disasters in their neighborhood. \\
They will likely be on FRS, GMRS, or MURS. \\
Ask around your neighborhood if anyone is CERT, if so find out what they are doing for communications and offer to help. If not encourage people to seek out the class (or take it yourself)
\end{frame}

\begin{frame}{Community Groups}
Beyond CERT find out if there are other community organized groups you could work with. Again check with them to find out what they use and how you can help
\end{frame}

\begin{frame}{Public Service Agencies}
Many public service agencies are going to be using LMR radios and frequencies. Fire, Police, and Search \& Rescue will all be using LMR.\\
Again find your local agencies and talk to them about what you can do 
\end{frame}

\begin{frame}{How do I find these People?}
I've mentioned several times about checking with the local groups, or agencies. You may be wondering how do I find out who they are?
\end{frame}

\begin{frame}{How do I find these People?}
The first place to check is your local Sheriff's office. Many of the CERT, ARES, and SAR groups will coordinate through the Sheriff.
\end{frame}

\begin{frame}{How do I find these People?}
The first place to check is your local Sheriff's office. Many of the CERT, ARES, and SAR groups will coordinate through the Sheriff. \\
Check ask the local government, nearby churches, and the Red Cross
\end{frame}

\begin{frame}{How do I find these People?}
The first place to check is your local Sheriff's office. Many of the CERT, ARES, and SAR groups will coordinate through the Sheriff. \\
Check ask the local government, nearby churches, and the Red Cross \\
The Internet can be your friend
\end{frame}

\begin{frame}{How do I find these People?}
The first place to check is your local Sheriff's office. Many of the CERT, ARES, and SAR groups will coordinate through the Sheriff. \\
Check with the local government, nearby churches, and the Red Cross \\
The Internet can be your friend \\
If you can't find one, Start one!
\end{frame}

\section{Ground Rules}

\begin{frame}{Ground Rules}
There are a few things to keep in mind before you get going.
\end{frame}

\begin{frame}{Check the attitude at the door}
You should be there to make things better. Leave the Ego behind, remember that there are always better ways of doing things and the more you ware willing to adapt the more useful you will be.
\end{frame}

\begin{frame}{It's the chain I go get and beat you with 'til ya understand who's in ruttin' command here.}
These also aren't the social clubs that you may be used to. There will be someone that you are ''working'' for and they might have a way of doing things that may be different from how you want to do things. \\
If you can't get along then someone will have to go, and if you can't figure out who that is going to be you shouldn't get involved in the first place.
\end{frame}

\begin{frame}{Utility}
Just because you know NTS, or FT8 doesn't mean that those skills and practices are going to have a 1:1 translation into the real world. \\
Be prepared to develop the skills you have and to learn new ones.
\end{frame}

\begin{frame}{Utility cont.}
Often as communicators we deal with those that are something else first. Firefighters, police, CERT, SAR, etc. will all have different lingo, and protocols. Don't get hung up on how things ''should'' be done and focus on getting them done, there are is a time and place to remind people what they are supposed to do, and a time to just let it be. 
\end{frame}

\section{EmComm}

\begin{frame}{EmComm}
Emergency communication is simply the handling of traffic in less than ideal situations
\end{frame}

\begin{frame}{Message Types}
Emergency communications can be broken down into a few different categories
\begin{description}
	\item[Formal Traffic] Official messages such as decelerations of Emergency
	\item[Tatical Traffic] ''Team Alpha go to the big 'X' on the map''
	\item[Staion Keeping] callsign, time, on the air, etc.
	\item[Health and Welfare] ''Cousin Bob is OK''
	\item[Garbage] ''Dude, there is this Jim guy just burning up all the finals''
\end{description}
\end{frame}

\begin{frame}{Message Types cont.}
	Each type of message has a place, though some shouldn't be on the air. There are each has it's own different set a protocols and techniques.
\end{frame}

\begin{frame}{Tools}
There will be different tools that work for different types of traffic. Use the right tool for the job and it will make your life easier. Use the wrong tool and not only with things be harder, but you will be waisting time.
\end{frame}

\begin{frame}{Winlink 2000}
Winklink is an email over radio tool, it really shines when you need to move lengthy and/or formal messages. It allows you to compose, or just copy and paste a message and to move that to any email address. There are even templates in place for many of the most used forms to make your life even easier. To give winlink justice would take its own presentation.
\end{frame}

\begin{frame}{ICS Forms}
The Incident Command System has it's own collection of forms, the more time you spend in formal EmComm the more you will deal with these forms. The 213 is the message form, useful for formal messages , the 309 is the comms log best used for tactical messages, and to track the movement of formal messages through a station.
\end{frame}

\section{Training}

\begin{frame}{Practice Practice Practice}
Practice makes perfect, the more you can take advantage of training opportunities the better. You should always come away with ways to improve
\end{frame}

\begin{frame}{This is the story of a fish}
As much as we think we may know it, a little humility goes a long way. It is better to admit our limitations than to over extend.
\end{frame}

\begin{frame}{Fake it till you make it}
Practice like it is for real, don't phone it in. Skipping steps in training is a good way to train people to skip steps. Those steps may be really important later on.
\end{frame}

\begin{frame}{FEMA NIMS/ICS}
The National Incident Management System and Incident Command System have trainings available, the ICS 100, 200, and 700 can be taken online. Some combination of these will required for most programs that deal with formal emergency response. 
\end{frame}

\begin{frame}{CERT \& SAR}
If you think you are going to communicate for either of these groups consider going through their training to see how they operate and get a view of what it is like on the other side.
\end{frame}

\section{Skills}

\begin{frame}{Relay}
Practice relaying for another station. This is a great example of go slow to go fast.
\end{frame}

\begin{frame}{Nets}
Being familiar with the procedures for checking in, passing traffic, and checking out of nets it vital
\end{frame}

\begin{frame}{Net Control}
More than just checking into a net, take a turn as net control. Try running a VHF, and an HF net they are a little different. \\ You may run one net for those in your neighborhood on FRS radios, then turn around and check into a higher level net to pass traffic
\end{frame}

\begin{frame}{QSY}
When two stations need to pass traffic between each other on a net, they should coordinate on the net, then generally QSY off frequency to pass the traffic. They can then return to the net frequency. All the while the net could be continuing to run.
\end{frame}

\begin{frame}{Write it Down!}
It is very, very important that when relaying or when moving traffic through a net to pass the message \textbf{VERBATIM!} Write things down and do \textbf{NOT} paraphrase or interpret what the message means, all the little details can be important. In the fire world there are Tankers, Tankers, and Tenders; if someone asks for a tender and you pass along that they want a tanker, you are going to have a bad day.
\end{frame}

\begin{frame}{Ask!}
As you get more practice you will start to get an idea of what is going on and what people might be looking for. If something doesn't make sense, ask the sender. A quick ''Just to verify you are asking for 'xyz'?'' doesn't hurt anything. They may have meant something other than what they said. Not only might this catch a mistake but lets the people you are dealing with know you are being attentive to the details.
\end{frame}

\begin{frame}{Practice, Practice, Practice. Again}
Make sure you regularly practice all the skills, and with your equipment. Software changes, radios may develop issues, and antennas may shift. Regularly stretching all the muscles makes sure that everything is ready to go when you need it. Don't try something for the first time when you need it to work.
\end{frame}

\section{Equipment}

\begin{frame}{Radios}
Talk about radios
\end{frame}

\begin{frame}{Power}
Keep those radios running
\end{frame}

\begin{frame}{Antennas}
That power has to go somewhere
\end{frame}

\begin{frame}{Computers}
Digital modes, logging etc
\end{frame}

\begin{frame}{Pen and Paper}
It helps to write stuff down
\end{frame}

\begin{frame}{Food and water}
The 'bear' necessities
\end{frame}

\begin{frame}{Comfort items}
A little comfort goes a long way, so I bring a chair
\end{frame}

%\section{Certifications}


\section{Fitness for Duty}

\begin{frame}{100\%}
If you are not 100 percent ready to do the job, then you could be making things worse. If nothing else you could be taking the spot of someone that is 100\%
\end{frame}

\begin{frame}{Impairment}
Drugs and Alcohol don't go well with an emergency, it is far better to sit out and sober up before taking part, you will only be a liability.
\end{frame}

\begin{frame}{Health}
If you are sick, or have other health issues that get in the way, be ready to take care of yourself. You may be able to power through a cold for a day, but does that one day make up for the next three you will be down for?
\end{frame}

\begin{frame}{Physical fitness}
Know what the job entails, and know your limits, and don't pretend to be more than you really are. You will only get yourself hurt. Find things you can do within your own limits, and be clear what those are.
\end{frame}

\section{Wrapping up}

\begin{frame}{Questions?}
Questions?
\end{frame}

\begin{frame}{The END}
The End.
\end{frame}
\end{document}